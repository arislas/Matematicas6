\chapter{Potencias y raíz cuadrada}

Las \textbf{potencias} son multiplicaciones de un número por él mismo cero o más veces. Tienen la forma $n^m$, donde a $n$ le llamamos \textbf{base} y a $m$ le llamamos \textbf{exponente}. La base $n$ es el número que tenemos que multiplicar y el exponente $n$, son las veces que tenemos que multiplicarlo.

\begin{itemize}
    \item En el caso especial en el que \textbf{elevamos un número a cero}, el resultado siempre es $1$.
    \item En el caso especial en el que \textbf{elevamos un número a uno}, el resultado siempre es el número que tenemos como base.
\end{itemize}

En el ejemplo siguiente puedes ver algunas potencias sencillas:

\begin{ejemplos}[label={Ejemplo:potencias}]{Algunas potencias sencillas}
    $2^0 = 1$ \\
    $2^1 = 2$ \\
    $2^2 = 2 \cdot 2 = 4$ \\
    $2^3 = 2 \cdot 2 \cdot 2 = 8$ \\
    $2^4 = 2 \cdot 2 \cdot 2 \cdot 2 = 16$ \\
    $2^5 = 2 \cdot 2 \cdot 2 \cdot 2 \cdot 2 = 32$ \\
    $\vdots$
\end{ejemplos}